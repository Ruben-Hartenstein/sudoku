%!TEX root = ../dokumentation.tex


\newcommand{\deAbstractContent}{
    % deutschen Abstract hier einfügen!

    Sudokus sind Logikrätseln und können in verschiedenen Medien gefunden werden. Es gibt unterschiedliche schwierigkeitsstufen von Sudokurästeln. Wie bei allen Logikrätseln kann ein Sudoku mit verschiedenen Lösungsstrategien gelöst werden. Die Lösungsstrategien können auch in ihrer Schwierigkeitsstufe variieren. 
    
 	In dieser Studienarbeit wird ein Web-Frontend entwickelt, dass einen User bei der Lösung eines schwierigen Sudokurätsels unterstützt. Zu Beginn kann ein User ein Rästels in ein leeres Gitter eingeben und es wird überprüft ob das Sudoku die Eigenschaft der eindeutigen Lösbarkeit besitzt. 
 	
 	Die Lösung des Sudoku soll mittles der Strategien Schritt für Schritt erklärt werden. Unteranderem soll die Technik benannt und anhand des konkreten Beispiels erläutert werden. Zudem soll auf dem Board der Bereich markiert werden indem der User eine technik anwenden kann. Diese Hilfestellungen werden in verschiedenen Abstufungen umgesetzt, sodass dem User erst die Chance gegeben wird eine Technik selbst anzuwenden, bevor die Erläuterung gegeben wird.
 	
 	In der Studienarbeit werden auch verschiedene technische Rahmenbedingungen wie Webserver oder Framework vealusiert und mathematischen Eigenschaften eines Sudokurätsels beleuchtet.
 
 	Die Implementierung der Studienarbeit soll in Python und JavaScript erfolgen. Eine weitere Anforderung ist eine benutzeroptimierte Darstellungen sowhol für Mobile Endgeräte als auch auf einem Desktop PC oder Laptop.
 	
 	
    
}

\newcommand{\enAbstractContent}{
    % englischen Abstract hier einfügen!
    \todo{english abstract....}
}

%%%%%%%% Ab hier nicht mehr anfassen! %%%%%%%%

\newcommand{\deAbstract}{%
    \renewcommand{\abstractname}{\langabstract} % Text für Überschrift
    \begin{abstract}
        \thispagestyle{plain}
        \deAbstractContent
    \end{abstract}
}

\newcommand{\enAbstract}{
    \renewcommand{\abstractname}{\langabstract} % Text für Überschrift
    \begin{abstract}
        \thispagestyle{plain}
        \enAbstractContent
    \end{abstract}
}

\iflang{de}{
    \deAbstract
    \ifbothabstracts
        \clearpage
      %  \begin{otherlanguage}{english}
            \enAbstract
     %   \end{otherlanguage}
    \fi
}

\iflang{en}{
    \enAbstract
    \ifbothabstracts
        \clearpage
        \begin{otherlanguage}{ngerman}
            \deAbstract
        \end{otherlanguage}
    \fi
}