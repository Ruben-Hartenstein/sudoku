%!TEX root = ../dokumentation.tex

\chapter{Fazit}

In diesem Kapitel wird abschließend geklärt ob die Forschungsaufgabe mit Erfolg beendet werden konnte. Zunächst werden die wichtigsten Designentscheidungen in kürze nochmals vorgestellt und daraufhin ein Fazit gezogen. Zum Abschluss wird hinausblicken nochmals auf den \textbf{Sudoku Solver} geschaut und weitere Verbesserungsmöglichkeiten genannt. 

\section{Ergebnisse}

In der Studienarbeit soll ein \textbf{Sudoku Solver} entwickelt werden. Dieser soll einem User über verschiedenen Abstufungen dabei helfen unterschiedliche Lösungstechniken auf einen Sudokurätsel anzuwenden und das Rätsel zu lösen. Dabei wird davon ausgegangen, dass ein Rätsel mit etwa 25 Lösungstechniken vollständig gelöst werden kann. 

In den Rahmenbedingungen werden die Programmiersprachen Python und JavaScript festgelegt. Neben den Programmiersprachen werden zwei Webserver verglichen, wobei am Ende entschieden wird, dass die Software auf einem Apache \Ac{HTTP} Server laufen soll und zwei Frameworks. Bei den Frameworks wurde sich für das leichtgewichtigere Flask entschieden, wobei das Entwurfsmuster des \ac{MVC} von Django übernommen wird.

Danach werden einige mathematische Eigenschaften von Sudokus beleuchtet, wobei die eindeutige Lösbarkeit von besonderer Relevanz ist. Denn wenn ein Sudoku nicht eindeutig Lösbar ist, dann können der \textbf{Sudoku Solver} und der User unterschiedliche Endergebnisse bekommen.

Beim Frontend wurde besonders auf die \ac{UX} geachtet sowie auf ein responsiv Design. Durch die Abstufungen der Hilfestellungen lässt sich bequem durch iterieren und das Frontend reagiert mit den jeweiligen Informationen darauf. Diese Informationen werden im Backend gesammelt. Über die Lösungstechniken, die alle identisch aufgebaut sind, weil sie von einer abstrakten Klasse erben, werden nacheinander von einfach bis schwiewirg auf den aktuellen Zustand des Boards ausprobiert. Insgesamt wurden 27 Lösungsstrategien implementiert.

\section{Beantwortung Forschungsfrage}

Als Fazit lässt sich sagen, dass die Aufgabe zum größten Teil umgesetzt wurde. Die meisten Anforderungen, wie ein responsive Design und das Implementieren von 25 Lösungsstrategien konnten umgesetzt werden. Mit den Hilfestellungen kann ein User die Anwendung der Techniken nachvollziehen und verstehen. Die Hilfestellungen teilen sich in unterschiedliche Stufen auf. Als erstes wird nur ein bestimmter Breich auf dem Board markiert und die Technik genannt, die angewendet werden kann. Im nächsten Schritt werden auch die relevanten Kandidaten in unterschiedlichen Farben hervorgehoben. Zuletzt bekommt der Nutzer ein auf das aktuelle Rätsel angepasste Erläuterung wie und vor allem warum eine Technik anwendbar ist.

Die Studienarbeit hat gezeigt, dass auch mit 27 Lösungsstrategien nicht jedes Sudoku gelöst werden kann. Die vorgestellten Ergebnisse werfen die weiterführende Fragen auf, ob eine andere Reihenfolge der Lösungstechniken zu einer Lösung führen könnten. Eine andere weiterführende Forschungsfrage wäre welche Techniken weiter implementiert werden müssten, damit jedes Sudokurätsel vom Sudoku Solver über logische und nachvollziehbare Methode gelöst werden könnten.

\section{Ausblick}

Der \textbf{Sudoku Solver} wurde zwar erfolgreich umgesetzt, trotzdem gibt es noch weiter Potential für Verbesserungen.

Ein Problem des \textbf{Sudoku Solver}s ist, dass trotz der Implementierung von 25 Lösungsstrategien gerade schwierige Sudokurätsel noch nicht von ihm gelöst werden können. Man könnte also noch weitere Lösungsstrategien aus anderen Quellen implementieren. Des weiteren ist die Reaktion darauf keine anwendbare Lösungstechnik zu finden nur suboptimal geregelt. Eine weitere Option wäre es dem User erst anzuzeigen, dass er keinen passenden Algorithmus für das Sudokurästel besitzt und daraufhin eine passende Zahl in das Gitter einfügt oder einen falschen Kandidaten eliminiert. Auch das Beenden des Sudokus könnte durch den optischen Effekt von Konfetti weiter zelebriert werden und das Board setzt sich nach einer bestimmten Zeit von alleine wieder zurück. 

Eine weitere spannende Ergänzung wäre das eigenständige Festlegen der Reihenfolge der Lösungstechniken. Damit könnte der User bestimmten Techniken eine höhere Priorität zuordnen, um eine Technik des öfteren zu finden, anzuwenden und zu verstehen. Dafür müssten in den Implementierungen der Lösungsstrategien einige Anpassungen vorgenommen werden, da aktuell zum Teil davon ausgegangen wird, dass bestimmte Konstellationen von Kandidaten bereits von einer früheren Technik gefunden werden.

Neben den Funktionalen Erweiterungen gibt es auch noch optisch Verbesserungsmöglichkeiten, da die Oberfläche in speziellen Situationen noch sehr leer aussieht oder größer sein könnte. Das \ac{UI} wurde zwar auf Basis des Empfinden der Entwickler benutzerorientiert gestaltet, mit weiterem Input des Kunden könnte jedoch weitere Optimierungen umgesetzt werden. 

