%!TEX root = ../dokumentation.tex

\chapter{Testing/Validierung}
Zum Testen und Validieren des \textbf{Sudoku Helper} wurden zwei verschiedenen Techniken angewandt. Das Frontend wurde mittels eines Akzeptanztests von den Entwicklern getestet und das Backend mit einem End-to-End-Test.

In diesem Kapitel wird kurz der jeweilige Test erläutert und dann die Umsetzung im \textbf{Sudoku Helper} beschrieben.

\section{Frontend}

Das Frontend wurde mittels einem Akzeptanztest überprüft. Hierbei findet eine Funktionsprüfung einer User Story statt. Dabei werden nicht funktionale Anforderungen wie die Benutzerfreundlichkeit und Leistung der Software getestet. Diese nicht funktionalen Anforderungen wie die Antwortzeit zwischen Front- und Backend oder das Verhalten der Oberfläche wurde früh in der Entwicklung durchgeführt.

Im Zuge dieses Tests wurden Fehler abgefangen die beispielsweise beim Neu laden der Website entstanden sind. In Meetings mit dem Kunden sind weitere Unstimmigkeiten aufgetreten, die aufgrund der Anmerkungen des Kunden verbessert wurden, wie beispielsweise die Auswahl der Farbgebung.

Ein abschließender Akzeptanztest sollte jedoch von dem Kunden der Software durchgeführt werden.

\section{Backend}

Im Backend muss die Korrektheit der Strategien überprüft werden. Diese Überprüfung findet während der Entwicklung statt. 

Zu den Erklärungen der Strategien sind visuelle Beispiele gegeben. Die Beispiele dienen als Grundlage für den Test einer Strategie. Wenn eine Strategie auf ein Beispiel angewendet werden kann, dann wird diese Strategie in weiteren Sudokus gesucht. Beim Finden wird von Entwicklern, mittels des analogen Anwendens überprüft inwiefern eine Strategie auch außerhalb der Grundlage korrekt angewandt wird. Diese Überprüfung findet im Optimalfall mehrere Male in unterschiedlichen Situationen, wie beispielsweise das Finden einer Strategie in verschiedenen Units, statt. Wenn eine Lösungsstrategie in unterschiedlichen Situationen von den Entwicklern nachvollzogen werden kann, so ist diese validiert.

\section{End-to-End Test}
Zu guter Letzt wurde ein End-to-End Test durchgeführt. In diesem Test wird nicht nur die Grundfunktionalität der Software überprüft, sondern auch die erweiterte Funktionalität im Falle von Grenzfällen wie zum Beispiel das Nichtfinden einer passenden Strategie. Damit wird der gesamte Prozess geprüft. 

Im Zuge des \textbf{Sudoku Helper}s wurde ein Sudoku eingegeben und die Hilfestellungen der Strategien auf ihre Richtigkeit überprüft. Auch in diesem Test abgedeckt ist die Reaktion des Frontend wenn im Backend keine passende Technik gefunden wird oder wenn das Sudoku korrekt gelöst wurde.

Mittels des Tests wird sichergestellt, dass alle Prozesse, in diesem Fall das Anwenden verschiedener Strategien, korrekt ausgeführt wird und die entsprechenden Informationen ans Frontend gelangen.