%!TEX root = ../dokumentation.tex
\lstset{
	frame=single,
	keywordstyle=\color{blue},
	commentstyle=\color{green},
	numbers=left,
}

\part{Implementierung}

\chapter{Frontend}

\section{\ac{UI}}
\todo{Responsiv scheiß}

\section{\ac{UX}}


\section{Abstufungen der Hilfestellungen}
Die Anforderung an die verschiedenen Abstufungen der Hilfestellung wird hauptsächlich über das Frontend geregelt. 

\subsection{Erste Hilfestellung}

\begin{lstlisting}[caption={Erste Hilfestellung}, label={lst:help0}]
	 socket.on('help0', function (technique_result) {
		$("#technique-name").text(technique_result['name']);
		$("#technique-explanation").text("");
		colorCells(technique_result['primaryCells'], 'rgb(255,216,115)');
		colorCells(technique_result['secondaryCells'], 'rgb(181,216,244)');
	});
\end{lstlisting}

\subsection{Zweite Hilfestellung}
\begin{lstlisting}[caption={Zweite Hilfestellung}, label={lst:help1}]
	socket.on('help1', function (technique_result) {
		if (!candidatesVisible) {
			candidatesVisible = !candidatesVisible;
			$('#candidate').css('color', 'red');
		}
		colorCandidates(technique_result['crossOuts'], 'red');
		colorCandidates(technique_result['highlights'], 'lime');
	});
\end{lstlisting}

\subsection{Dritte Hilfestellung}
\begin{lstlisting}[caption={Dritte Hilfestellung}, label={lst:help2}]
	socket.on('help2', function (technique_result) {
		$("#technique-explanation").text(technique_result['explanation']);
	});
\end{lstlisting}

\subsection{Vierte Hilfestellung}
\begin{lstlisting}[caption={Vierte Hilfestellung}, label={lst:help3}]
	socket.on('help3', function () {
		if (!candidatesVisible) {
			candidatesVisible = !candidatesVisible;
			$('#candidate').css('color', 'red');
		}
		$("#technique-name").text("");
		$("#technique-explanation").text("");
		resetCellColor();
	});
\end{lstlisting}

\section{Fertig gelöstes Sudoku}