%!TEX root = ../dokumentation.tex

\chapter{Einleitung}
In vielen Zeitungen ist eine Vielzahl von Logikrätseln gedruckt. Darunter eines der häufigsten Logikrätsel: das Sudoku.
In der üblichen Version ist es das Ziel, ein 9×9-Gitter mit den Ziffern 1 bis 9 zu füllen. Dabei kann das Gitter in drei unterschiedliche Einheiten aufgeteilt werden. Diese Einheiten sind die Spalten, Zeilen und Blöcke des Gitters. Ein Block ist ein 3×3-Unterquadrat des Gitters. In jeder Einheit darf jede Ziffer genau einmal vorkommen.
Ausgangspunkt ist ein Gitter, in dem bereits mehrere Ziffern vorgegeben sind, damit das Sudoku eindeutig lösbar ist. Sudokus gibt es in unterschiedlichen Schwierigkeitsgraden. 
 

\section{Motivation}
Sudokus gibt es mittlerweile in vielen verschiedenen Varianten, mit modifizierten Regeln. In dieser Arbeit und für den \textbf{Sudoku Helper} wird jedoch nur das klassische Sudoku, wie es in der Einleitung beschrieben wurde, betrachtet. Es gibt verschiedene Schwierigkeitsstufen von Sudokus. Meistens werden Sudokus mit vielen vorgegebenen Zahlen als einfach eingestuft und Sudokus mit wenigen vorgegebenen Zahlen als schwierig. Ein weiterer Punkt, der Einfluss auf die Schwierigkeitsstufe eines Sudokus hat, ist die Anordnung der vorgegebenen Ziffern.

Wer schon einmal versucht hat ein schweres Sudoku  zu lösen, ist vielleicht auch bereits an seine Grenzen gekommen. Gerade bei schwierige Sudokus mit wenigen vorgebenden Zahlen kann es schwierig sein selbst eine passende Lösungsstrategie für das Sudoku zu finden. Oft sind anwendbare Lösungsstrategien schwierig zu erkennen und es lassen sich damit meist nur Kandidaten eliminieren. Einfachen Sudokus lassen sich oft durch simple logische Schlussfolgerungen lösen und Lösungsstrategien werden meist unbewusst und intuitiv eingesetzt. 

In dieser Studienarbeit soll ein \textbf{Sudoku Helper} entwickelt werden, der Lösungsstrategien programmatisch erkennt und dem Benutzer bei dem Finden eines Lösungsschrittes unterstützt. Diese Unterstützung erfolgt dabei in mehreren Stufen, von der Visualisierung, über die Erklärung bis zur automatischen Umsetzung der Strategie.


\section{Forschungsstand}
Im Internet findet man immer wieder Websites mit dem Angebot Sudokus zu lösen. Hierbei können die angegebenen Zahlen eingegeben werden und das Sudoku wird korrekt vervollständigt, falls es eine eindeutige Lösung gibt. Dabei bekommt der Nutzer nur das gelöste Sudoku ohne Informationen darüber, wie man ein Sudoku als Mensch ohne die Rechenleistung eines Computers logisch lösen kann. Viele Websites erklären bereits, welche Lösungsstrategien existieren, wie sie funktionieren und wie sie sich anwenden lassen, nutzen dabei aber meist konkrete Beispiele.

%Das Beherrschen der Lösungsstrategien besitzt einen großen Stellenwert. Einfache Sudokus lassen sich meist noch intuitiv und durch konzentriertes Absuchen lösen. Die beiden dafür ausreichenden %Anfängerstrategien heißen „Nackter Single“ und unter „Versteckter Single“ und werden im Verlauf dieser Arbeit noch erläutert. Sie werden mehrheitlich intuitiv und unbewusst eingesetzt, wenn für ein bestimmtes %Feld nur eine Zahl möglich ist oder wenn eine bestimmte Zahl nur in ein einziges Feld passt. Bei kniffligeren Sudokus müssen gewissenhaft Notizen gemacht werden. Dafür wird die Methode der „Kandidaten“ %verwendet. Damit lassen sich die Zusammenhänge am besten beschreiben. Manche Situationen in einem anspruchsvollen Sudoku erfordern sehr komplexe Lösungsansätze, die nicht immer so einfach zu verstehen sind. 


\section{Forschungsfrage}
Für diese Studienarbeit ist die Zielsetzung, ein \textbf{Sudoku Helper} zu entwickeln. Ziel ist es nicht ein Sudoku automatisch zu lösen, sondern dem Benutzer Schritt für Schritt den Weg zur Lösung aufzuzeigen. Auf dieser Basis wird zunächst geprüft, ob das Sudoku eindeutig lösbar ist. Ist dies der Fall, muss der Nutzer die Möglichkeit haben sich Anweisungen geben zu lassen, wie man weiter vorgehen könnte und welche Lösungsstrategie der Benutzer anwenden kann, damit man beim lösen des Sudokus fortschreiten kann. Auf Basis der bereits eingetragenen Zahlen und den übrig gebliebenen Kandidaten, soll geprüft werden, welche Strategie momentan anwendbar ist. Dem Benutzer soll daraufhin die anwendbare Strategie genannt und anhand des konkreten Sudokus erklärt werden, damit der Benutzer das Sudoku Lösen kann. Diese Anweisungen, im folgenden auch Hilfestellungen genannt, sind in vier Varianten unterteilt.

Für die erste Hilfestellung soll das Programm dem Benutzer die anzuwendende Strategie nennen und den Bereich hervorheben, auf dem die Strategie angewendet werden soll. Wenn diese Hilfestellung nicht ausreichend ist, dann markiert das Programm die konkreten Zellen und Kandidaten, die von der Strategie betroffen sind. Mit der dritten Hilfestellung soll eine Erläuterung gezeigt werden, warum bestimmte Kandidaten laut Strategie gestrichen werden können. Mit der letzten Hilfestellungen löscht das Programm den betroffenen oder mehrere betroffene Kandidaten. Für Ausnahmen, in denen eine Strategie eine Zahl hervorhebt, die eingetragen werden kann, erfolgt das Eintragen der Zahl anstatt das Löschen von Kandidaten.

Um ein eindeutig lösbares Sudoku vervollständigen zu können, soll der \textbf{Sudoku Helper} etwa 25 Lösungsstrategien beherrschen.

Das Programm soll auf einem Ubuntu Server mit Apache und Django laufen. Des Weiteren gibt es zusätzlich die Anforderung, dass das Web-Frontend responsive sein soll und immer eine benutzeroptimierte Darstellungen liefert.